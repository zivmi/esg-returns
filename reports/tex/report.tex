\documentclass[12pt, a4paper]{article}%
\usepackage[utf8]{inputenc}
\usepackage[english]{babel}
\usepackage{graphicx}
\usepackage{amssymb}
\usepackage{amsmath}
\usepackage{geometry}

%\setlength{\textheight}{24cm}
%\setlength{\textwidth}{160mm}
%\setlength{\topmargin}{-1.5cm}		
%\setlength{\evensidemargin}{-0mm}
%\setlength{\oddsidemargin}{-0mm}

%\setlength{\voffset}{-20mm}
%\setlength{\hoffset}{-7mm}

\usepackage{hyperref} %Must be loaded at the end.


\begin{document}

%---------------------------------------------------
%
%%%%%%%%%%%%%%%%%%% front page %%%%%%%%%%%%%%%%%%%%%%%
%
%---------------------------------------------------

\begin{titlepage}

\setlength{\topmargin}{0.5cm}
% \setlength{\bottommargin}{0.1cm}

\center

{\Large \bfseries Do ESG factors relate to yields of companies?
}\\[0.5cm] 

Mariia Kuzmina \footnote{\texttt{mariia.kuzmina@uzh.ch}},
Timon Gehrig \footnote{\texttt{timon.gehrig@bf.uzh.ch}},
Jennifer Li \footnote{\texttt{jennifer.li@execed.uzh.ch}},
Miroslav Zivanovic \footnote{\texttt{miroslav.zivanovic@uzh.ch}}\\
\today
\\ [2cm]

\begin{abstract}
ESG returns are awesome test test 
\end{abstract}

\vspace{3cm}



\vfill 
\end{titlepage}

\tableofcontents


%---------------------------------------------------
%
%%%%%%%%%%%%%%%%%%% Intro %%%%%%%%%%%%%%%%%
%
%---------------------------------------------------
\newpage
\section{Introduction}
In this research project, we assess the factors introduced by \textcite{FamaFrench1992}. Additionally, we incorporate Environmental, Social, and Governance [ESG] factors into the factor models.
The question arises whether these are valuable factors. Based on this, we examine following research question:\\


How do ESG factors relate to yields of companies?

%---------------------------------------------------
%
%%%%%%%%%%%%%%%%%%%%%% Main sections %%%%%%%%%%%%%%%%%%%%%%%
%												   
%---------------------------------------------------
\section{Factor Models and ESG in existing research}
There exist various approaches to modeling asset prices. One central theory backed by the assumption of investor's rationality is the Efficient Market Hypothesis. It states that all information about a firm value is reflected by its stock price. Accordingly, generating excess returns in terms of alpha is not possible (\textcite{Fama1970}). 

Based upon this, \textcite{Sharpe1964} and \textcite{Lintner1965} introduced the Capital Asset Pricing Model [CAPM]. Given the systematic risk of a stock or a portfolio, the expected return can be calculated. Respectively, the only factor is the beta which represents the market risk. This inevitably leads to anomalies caused by the model, suggesting the need for more factors.

As an alternative, the Arbitrage Pricing Theory [APT] was developed by \textcite{Ross1976} which loosens the assumption that markets are perpetually efficient. Instead, security prices may deviate from the fair value occasionally. However, these temporary mispricings are eventually corrected by market action, making arbitrage opportunities only possible in the short term.
The multi-factor asset pricing models use the idea that there is a linear relationship between the macroeconomic factors and the expected return.

\textcite{FamaFrench1992} have proposed some factors based on the anomalies that were noticed in the CAPM.
First of all, Small-Cap stocks seemingly yield higher expected returns compared to the market which justifies the extension of a size factor called Small Minus Big [SMB].

Additionally, the value anomaly was observed where stocks with a high book-to-market ratio tend to perform better than the market, introducing the High Minus Low [HML] factor.
This leads to the Fama and French Three Factor Model with the two additional factors SMB and HML.

In recent years, ESG factors have gained significant attention in the financial landscape and has emerged as a key determinant of a company’s financial performance. Fama and French's Three-Factor model, which considers risk factors including market risk, size, and value to predict returns, has been integral to yield analysis. To this model, we integrate ESG elements.
Correlating ESG factors with yields is sensible, given that companies with superior ESG standards may achieve higher operational efficiency, establish better stakeholder relationships, and move towards overall risk reduction. These factors could contribute to a stronger financial profile, and by extension, higher yields. Conversely, companies displaying inadequate ESG measures could face operational risks and regulatory sanctions, leading to lower yields.

\section{Methodology} % Check if it's correct
The time period that is observed ranges from September 1st, 2014 to September 1st, 2022. Based on that, the constituents of Nasdaq 100 are fetched. This is done using Wikipedia as the source.
The \textcite{FamaFrench1992} factors are obtained from the official website. 
Lastly, monthly returns are fetched for the specific index. \\% Maybe write one more sentence about it?

Using these data sets, columns consisting with only NA values are dropped, while imputation is implemented using the fill-forward method. % Correct?

\subsection{Model}
Using the approach based on the \textcite{FamaFrench1992} procedure, the 3-factor model is used. We thus obtain following regression equation:
\begin{equation} \text{E}(\text{r}_{\text{i}}) = \text{r}_{\text{f}} + \beta \cdot (\text{r}_{\text{M}} - \text{r}_{\text{f}}) + \beta_{\text{SMB}} \cdot \text{SMB} + \beta_{\text{HML} \cdot \text{HML} \end{equation}

Additionally, a multiple linear regressions is performed where monthly returns are regressed on the factors of the 3-factor model, together with a factor which respresents the ESG aspects. It combines the E-, S-, and G-Scores. Accordingly, we refer to it as the ESG factor.

\begin{equation} \text{E}(\text{r}_{\text{i}}) = \text{r}_{\text{f}} + \beta \cdot (\text{r}_{\text{M}} - \text{r}_{\text{f}}) + \beta_{\text{SMB}} \cdot \text{SMB} + \beta_{\text{HML} \cdot \text{HML} + \beta_{\text{ESG}} \cdot \text{ESG} \end{equation}

Based on those two equations, we obtain following summery statistis: \\
% Add summary statistics: FF3FM

\usepackage{tabularray}
\begin{table}
    \centering
    \begin{tblr}{
    }
                 & Intercept          & Market Risk Premium & SMB & HML & ESG \\
    Coefficients & -5.900 * 10\^{-2} & -8.300 * 10\^{-5}     & -6.167 * 10\^{-3}    & -3.050 * 10\^{-3}     & 1.009 * 10\^{-3}    \\
    t-Value      & 0.068                   & -0.968                    & 0.432    & -0.642    & 0.010    \\    
    \end{tblr}
    \end{table}



Further, we get plots displaying the intercept, coefficient and the t-statistic.\\ 
% Add plots: FF3FM

% Add plots: FF3FM + ESG

\section{Results} 
% Add about FF3FM

Once the ESG factor is included in the factor model, the dynamics are as follows: The market risk premium has a negative regression slope, indicating that there does not exist a premium rewarding systematic risk. The risk-return profile can not be used to get high returns.
The coefficients for SML and HML are both negative, which would mean that the optimal investment strategy is buying stocks issued by large firms, as well as growth stocks.
The only factor exhibiting a positive coefficient is the ESG factor. Hence, high ESG ratings can lead to outperformance.\\
However, as none of the t-values are statistically significant under the 5\% significance level, these results must be handled with caution.
To increase statistical significance, one should increase the inspected time-period. Further, by increasing the sample size, the statistical significance might increase.


\begin{thebibliography}{1} % Does this compile correctly?

%\bibitem{Paolellalinear}
%Marc S. Paolella (2018) \emph{Linear Models and Time-Series Analysis}, John Wiley and Sons

@article{Fama1970,
    title={Efficient Capital Markets: A Review of Theory and Empirical Work},
    author={Fama, Eugene F.},
    journal={The Journal of Finance},
    volume={25},
    number={2},
    pages={384-385},
    year={1970},
    publisher={Wiley for the American Finance Association}
}

@article{Sharpe1964,
    title={Capital Asset Prices: A Theory of Market Equilibrium under Conditions of Risk},
    author={Sharpe, William F.},
    journal={The Journal of Finance},
    volume={19},
    number={3},
    pages={425-442},
    year={1964},
    publisher={Wiley for the American Finance Association}
}

@article{Lintner1965,
    title={The Valuation of Risk Assets and the Selection of Risky Investments in Stock Portfolios and Capital Budgets},
    author={Lintner, John},
    journal={The Review of Economics and Statistics},
    volume={47},
    number={1},
    pages={13-37},
    year={1965},
    publisher={MIT Press}
}

@article{Ross1976,
    title={The Arbitrage Theory of Capital Asset Pricing},
    author={Ross, Stephen A.},
    journal={Journal of Economic Theory},
    volume={13},
    number={3},
    pages={341-360},
    year={1976},
    publisher={Elsevier}
}

@article{FamaFrench1992,
    title={The Cross-Section of Expected Stock Returns},
    author={Fama, Eugene F. and French, Kenneth R.},
    journal={The Journal of Finance},
    volume={47},
    number={2},
    pages={427-465},
    year={1992},
    publisher={Wiley for the American Finance Association}
}


@article{FamaFrench2014,
    title={A five-factor asset pricing model},
    author={Fama, Eugene F. and French, Kenneth R.},
    journal={Journal of Financial Economics},
    volume={116},
    number={1},
    pages={1-22},
    year={2014},
    publisher={Elsevier}
}


@article{FamaFrench2012,
  title = {Size, Value, and Momentum in International Stock Returns},
  author = {Fama, Eugene F. and French, Kenneth R.},
  journal = {Journal of Financial Economics},
  volume = {105},
  number = {3},
  pages = {457-472},
  year = {2012},
  publisher = {Elsevier},
}


@article{FamaFrench1997,
    title = {Industry costs of equity},
    author = {Fama, Eugene F. and French, Kenneth R.},
    journal = {Journal of Financial Economics},
    volume = {43},
    number = {2},
    pages = {153-193},
    year = {1997},
}


\end{thebibliography}


\end{document}
